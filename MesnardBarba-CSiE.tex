\documentclass[10pt,twocolumn,twoside]{article}

\usepackage{palatino}  % use 10pt size with Palatino.
\usepackage[T1]{fontenc}

\usepackage[left=3.5cm,top=2.54cm,right=2.5cm,bottom=2.8cm,nohead]{geometry} % load first
\usepackage{amsmath}
\usepackage[font=sf]{caption}
\usepackage{color}
\usepackage{fancyhdr}
\usepackage{graphicx}
\usepackage{lettrine}
\usepackage[super,sort&compress]{natbib}
\usepackage{pifont}
\usepackage{sectsty}
\usepackage{setspace}
\usepackage{sidecap}  % allows side captions for figures
\usepackage{subfig}
\usepackage{wrapfig}

\lhead{}\rhead{}
\fancyfoot{} % clear all footer fields
\fancyfoot[LE,RO]{\thepage}
\renewcommand{\headrulewidth}{0pt}
\renewcommand{\footrulewidth}{0.4pt}

\definecolor{sectcol}{rgb}{0.0,0.24,0.43} % {0,0,0}
\definecolor{dropped}{rgb}{0.55,0.06,0.11}
\definecolor{update}{rgb}{0.098,0.357,0.675} % ------------ REVISED TEXT COLOR, change to {0,0,0}

\setcounter{DefaultLines}{4}
\renewcommand{\DefaultLoversize}{0.05}

\renewcommand{\captionlabelfont}{\bf\sffamily}
\renewcommand{\captionfont}{\sffamily\footnotesize}
\sectionfont{\large\sffamily\color{sectcol}\vspace{-2mm}}
\subsectionfont{\normalsize\sffamily\bfseries\vspace{-2mm}}
\renewcommand{\bibfont}{\sffamily\footnotesize}

\newlength{\up}
\setlength{\up}{-2.5mm}

% hyperref package should always be loaded as the very last one
% to be sure that it has the last word ...
\usepackage[
    pdftex,
    pdftitle={},
    pdfauthor={L.~A. Barba},
    pdfpagemode={UseOutlines},
    bookmarks, bookmarksopen,bookmarksnumbered={True},
    colorlinks, linkcolor={black},citecolor={black},urlcolor={black}
    ]{hyperref}
    
%\title{\Huge{\color{sectcol}Fast \textit{N}-body Simulations on GPUs}}
%
%\author{ \sf \textbf {Rio Yokota, L.~A. Barba}\\
%\sf\normalsize  Mechanical Engineering Department, Boston University, Boston MA 02215}
%\date{}
%
%\begin{document}
%\maketitle

\begin{document}
\pagestyle{fancy}

\twocolumn[ %this produces full-width text on a twocolumn document
{\Huge{\color{sectcol}Title}}
\vspace{0.8cm}

{ \sf 
\onehalfspacing
{\large \textit{Short summary}}
\vspace{1cm}

\textbf {Olivier Mesnard, Lorena A. Barba}\\
  Mechanical and Aerospace Engineering, George Washington University, Washington DC 20052

\vspace{1cm}
}
] %close full-width text


\lettrine{\textcolor{dropped}{R}}{}eproducibility

%\section*{}
%\vspace{\up}

%\begin{figure}
%\centering
%\includegraphics[width=0.49\textwidth]{figs/figurefile.pdf}
%\caption{Caption.}
%\label{fig:template}
%\end{figure}

\vspace{2cm}

Our codes are available for unrestricted use, under the MIT license; to obtain the codes and run the tests in this paper, the reader may follow instructions on ...



\section*{Acknowledgements}
\vspace{\up}

{\sf \emph{We're grateful for the support from the US National Science Foundation and the Office of Naval Research. Recent grant numbers are NSF OCI-0946441, and ONR award \#N00014-11-1-0356.}

\bigskip

}

\bibliographystyle{unsrt}
\bibliography{scicomp}

\vspace{1cm}

\small
{\sf 


\noindent \textbf{Olivier Mesnard} bio
\bigskip

\noindent\textbf{Lorena A. Barba} is Associate Professor of Mechanical and Aerospace Engineering at the George Washington University.
She obtained her PhD in Aeronautics from the California Institute of Technology.
Her research interests include computational fluid dynamics, especially particle methods for fluid simulation and immersed boundary methods; fundamental and applied aspects of fluid dynamics, especially flows dominated by vorticity dynamics; the fast multipole method and applications; and scientific computing on GPU architecture. 
She received the Amelia Earhart Fellowship, a First Grant award from the UK Engineering and Physical Sciences (EPSRC), the National Science Foundation Early CAREER award, and was named CUDA Fellow by NVIDIA, Corp.
}


\end{document}